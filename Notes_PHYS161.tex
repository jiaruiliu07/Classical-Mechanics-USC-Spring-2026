\documentclass[10pt]{article}
\usepackage{NotesTeXV3,lipsum}
\usepackage{lmodern}
\usepackage{fontspec}
\usepackage{physics}

% Physics note color palette
\definecolor{physgreen}{RGB}{34,139,34}
\definecolor{physpurple}{RGB}{128,0,128}
\definecolor{physblue}{RGB}{30,144,255}
\definecolor{physorange}{RGB}{255,140,0}
\definecolor{physgray}{RGB}{80,80,80}

\setmainfont{Source Serif Pro}


\tolerance=10000
\emergencystretch=3em
\hbadness=10000
\hfuzz=2pt


\begin{document}
	\title{{Advanced Principles of Mechanics I}\\{\normalsize{Notes on USC Spring 2026 Course PHYS-161}}}
	\author{Jiarui Liu}
	\affiliation{
	High School Student at Nanjing Foreign Language School\\
	\href{https://conversionindex.org}{Website}\\
	\href{https://www.linkedin.com/in/jiarui-liu-994007294/}{LinkedIn}\\
	\href{https://github.com/jiaruiliu07}{GitHub}\\
	}
	\emailAdd{jliu1978@usc.edu}
	\maketitle
	\newpage
	\pagestyle{fancynotes}
	\part{Special Relativity}

	\section{Dimensions and Dimensional Analysis}
	\textit{This will be an experimental course: starting with the concept of space-time, going back to kinematics, then relativistic dynamics, lastly energy.}
	
	\vspace{1\baselineskip}
	\noindent Definition of physics:
	\begin{definition}
		Physics is the natural quantitative science that deals with the fundamental interactions between matter and energy.
	\end{definition}

	What arguably sets physics apart from the other sciences would be the reductionist nature. At the base of the "pyramid of the natural sciences" lies mathematics, the fundamental subjects. Up a level is Physics, then Chemistry, then Biology. Because it's quantitative, when we study the world around us with measurements and standardizations. This introduces us to \textit{dimensions and unit}:
	\begin{claim}
		Nearly all properties in classical physics can be described with one or more of the three dimensions: length, time, and mass.
	\end{claim}

	The notation would be $[x]$ as the dimension of some variable or constant, for example
	\begin{equation}
	  [x] = L.
	\end{equation}

	The SI unit is the international unit. Remember that the standard for mass in SI is kilograms. Now we introduce the natural unit. 

	\begin{definition}
		The natural unit is defined by setting the values of the three constants:
		\begin{equation}
	  	  c = \hbar = G = 1.
		\end{equation}
	\end{definition}

	Here they still have dimensions, for example:
	\begin{equation}
	  [c] = \frac{L}{T} = 1 \quad \Rightarrow \quad L = T.
	\end{equation}

	\noindent \textbf{Dimensional Analysis}

	\vspace{1\baselineskip}
	A simple pendulum on Earth has length $L$, mass $m$. Given
	\begin{equation}
	  [l] = L, [m] = M, \;\text{and}\; [g] = \frac{L}{T^{2}}
	\end{equation}

	determine, up to dimensionless factors, the period of the pendulum.

	\vspace{1\baselineskip}
	\textbf{Solution.} We have, apparently, the period $\tau$ with dimension $T$. We would assume the powers with greek letters:
	\begin{equation}
	  [\tau] = T = [l^{\alpha} m^{\beta} g^{\gamma}] = L^{\alpha} M^{\beta} \left(\frac{L}{T^{2}}\right)^{\gamma}.
	\end{equation}

	This is solved simply as:
	\begin{equation}
	  \begin{cases} \alpha + \gamma = 0, \\ \beta = 0, \\ -2\gamma = 1 \end{cases} \quad \Rightarrow \quad \begin{cases} \alpha = \frac{1}{2}, \\ \beta = 0, \\ \gamma = -\frac{1}{2} \end{cases}.
	\end{equation}

	Which gives us
	\begin{equation}
	  \tau \propto l^{\frac{1}{2}} g^{-\frac{1}{2}} = \sqrt{\frac{l}{g}}
	\end{equation}

	EXERCISE: Look at the dimensions of $c,\hbar, \,\text\,{and} G$. Create the Planck units- the Planck length, time, and mass, by combining the three constants in a unique way that they create a constant for length, time, and mass.

	
	\section{Definitions}
	We start by defining the basics of kinematics- the study of "how things move". The simplest case of motion is none- an object being stationary. Now there will be a question: where is it. Now we define the frame of reference and point particles. 
	\begin{definition}
		A frame of reference for an observer is a sufficiently precise laboratory of measurement approaches, i.e. clocks, meter sticks. Here, in classical physics, we assume a precise measurement without quantum fluctuation, and it can be arbitrarily precise.
	\end{definition}

	And we define the point particle:
	\begin{definition}
		A point particle is an approximation whereby an object is treated as a zero-dimensional, mathematical point. It is valid when the size and shape of the object are irrelevant for the question under study.
	\end{definition}

	In this frame of reference, we may set up Cartesian coordinates for space $(x,y,z)$ to uniquely specify the position of a particle. The Cartesian coordinate is defined by three everywhere mutually perpendicular axes intersecting at the same point, defined as the origin. 
	
	\vspace{\baselineskip}
	Now it's the problem of incorporating time: the sequence of locations a particle passes can be described as time. It cannot, to our best knowledge be reversed, so it's the monotonically increasing parameter denoting the particle's position. We treat it as another label on the event: write $(t,x,y,z)$. Thus, to locate an event in space, we have four dimensions. 

	\begin{definition}
		Hi.
	\end{definition}
\end{document}